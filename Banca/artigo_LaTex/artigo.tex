%% abtex2-modelo-artigo.tex, v-1.7.1 laurocesar
%% Copyright 2012-2013 by abnTeX2 group at http://abntex2.googlecode.com/ 
%%
%% This work may be distributed and/or modified under the
%% conditions of the LaTeX Project Public License, either version 1.3
%% of this license or (at your option) any later version.
%% The latest version of this license is in
%%   http://www.latex-project.org/lppl.txt
%% and version 1.3 or later is part of all distributions of LaTeX
%% version 2005/12/01 or later.
%%
%% This work has the LPPL maintenance status `maintained'.
%% 
%% The Current Maintainer of this work is the abnTeX2 team, led
%% by Lauro César Araujo. Further information are available on 
%% http://abntex2.googlecode.com/
%%
%% This work consists of the files abntex2-modelo-artigo.tex and
%% abntex2-modelo-references.bib
%%

% ------------------------------------------------------------------------
% ------------------------------------------------------------------------
% abnTeX2: Modelo de Artigo Acadêmico em conformidade com
% ABNT NBR 6022:2003: Informação e documentação - Artigo em publicação 
% periódica científica impressa - Apresentação
% ------------------------------------------------------------------------
% ------------------------------------------------------------------------

\documentclass[
	% -- opções da classe memoir --
	article,			% indica que é um artigo acadêmico
	11pt,				% tamanho da fonte
	oneside,			% para impressão apenas no verso. Oposto a twoside
	a4paper,			% tamanho do papel. 
	% -- opções da classe abntex2 --
	%chapter=TITLE,		% títulos de capítulos convertidos em letras maiúsculas
	%section=TITLE,		% títulos de seções convertidos em letras maiúsculas
	%subsection=TITLE,	% títulos de subseções convertidos em letras maiúsculas
	%subsubsection=TITLE % títulos de subsubseções convertidos em letras maiúsculas
	% -- opções do pacote babel --
	english,			% idioma adicional para hifenização
	brazil,				% o último idioma é o principal do documento
	]{abntex2}


% ---
% PACOTES
% ---

% ---
% Pacotes fundamentais 
% ---
\usepackage{cmap}				% Mapear caracteres especiais no PDF
\usepackage{lmodern}			% Usa a fonte Latin Modern
\usepackage[T1]{fontenc}		% Selecao de codigos de fonte.
\usepackage[utf8]{inputenc}		% Codificacao do documento (conversão automática dos acentos)
\usepackage{indentfirst}		% Indenta o primeiro parágrafo de cada seção.
\usepackage{nomencl} 			% Lista de simbolos
\usepackage{color}				% Controle das cores
\usepackage{graphicx}			% Inclusão de gráficos
% ---
		
% ---
% Pacotes adicionais, usados apenas no âmbito do Modelo Canônico do abnteX2
% ---
\usepackage{lipsum}				% para geração de dummy text
% ---
		
% ---
% Pacotes de citações
% ---
\usepackage[brazilian,hyperpageref]{backref}	 % Paginas com as citações na bibl
\usepackage[alf]{abntex2cite}	% Citações padrão ABNT
% ---

% ---
% Configurações do pacote backref
% Usado sem a opção hyperpageref de backref
\renewcommand{\backrefpagesname}{Citado na(s) página(s):~}
% Texto padrão antes do número das páginas
\renewcommand{\backref}{}
% Define os textos da citação
\renewcommand*{\backrefalt}[4]{
	\ifcase #1 %
		Nenhuma citação no texto.%
	\or
		Citado na página #2.%
	\else
		Citado #1 vezes nas páginas #2.%
	\fi}%
% ---

% ---
% Informações de dados para CAPA e FOLHA DE ROSTO
% ---
\titulo{Artigo Científico Pacific Transit}
\autor{Gabriel Vieira Figueiredo Tomaz}
\local{Brasil}
\data{2013, 28 Novembro}
% ---

% ---
% Configurações de aparência do PDF final

% alterando o aspecto da cor azul
\definecolor{blue}{RGB}{41,5,195}

% informações do PDF
\makeatletter
\hypersetup{
     	%pagebackref=true,
		pdftitle={\@title}, 
		pdfauthor={\@author},
    	pdfsubject={Artigo Pacific Transit},
	    pdfcreator={LaTeX with abnTeX2},
		pdfkeywords={abnt}{latex}{abntex}{abntex2}{atigo científico}, 
		colorlinks=true,       		% false: boxed links; true: colored links
    	linkcolor=blue,          	% color of internal links
    	citecolor=blue,        		% color of links to bibliography
    	filecolor=magenta,      		% color of file links
		urlcolor=blue,
		bookmarksdepth=4
}
\makeatother
% --- 

% ---
% compila o indice
% ---
\makeindex
% ---

% ---
% Altera as margens padrões
% ---
\setlrmarginsandblock{4cm}{4cm}{*}
\setulmarginsandblock{4cm}{4cm}{*}
\checkandfixthelayout
% ---

% --- 
% Espaçamentos entre linhas e parágrafos 
% --- 

% O tamanho do parágrafo é dado por:
\setlength{\parindent}{1.3cm}

% Controle do espaçamento entre um parágrafo e outro:
\setlength{\parskip}{0.2cm}  % tente também \onelineskip

% Espaçamento simples
\SingleSpacing

% ----
% Início do documento
% ----
\begin{document}

% Retira espaço extra obsoleto entre as frases.
\frenchspacing 

% ----------------------------------------------------------
% ELEMENTOS PRÉ-TEXTUAIS
% ----------------------------------------------------------

%---
%
% Se desejar escrever o artigo em duas colunas, descomente a linha abaixo
% e a linha com o texto ``FIM DE ARTIGO EM DUAS COLUNAS''.
% \twocolumn[    		% INICIO DE ARTIGO EM DUAS COLUNAS
%
%---
% página de titulo
\maketitle

% resumo em português
\begin{resumoumacoluna}

O Projeto Interativo II trata do desenvolvimento de um jogo em linguagem C, em conjunto com a biblioteca
Allegro 5, com tema livre e que 
cujo objetivo principal é ter um fundo educacional.
Nesse trabalho, o tema escolhido foi a educação e conhecimento sobre o trânsito,
tema muito recorrente, uma vez que está presente no nosso dia-a-dia.\\
 
 \vspace{\onelineskip}
 
 %\noindent
 %\textbf{Palavras-chaves}: latex. abntex. editoração de texto.
\end{resumoumacoluna}

% ]  				% FIM DE ARTIGO EM DUAS COLUNAS
% ---

% ----------------------------------------------------------
% ELEMENTOS TEXTUAIS
% ----------------------------------------------------------
\textual

% ----------------------------------------------------------
% Introdução
% ----------------------------------------------------------
\section*{Introdução}

A aplicação de temas educativos a jogos digitais tem grandes vantagens a métodos de ensino comuns, pois
jogos podem se tornar fonte de entusiasmo desde crianças até adultos, conseguindo manter os participantes
em um alto grau de concentração e envolvimento, assim participando ativamente das tarefas e desafios 
propostos pelo jogo.

"Acredita-se que transportando jogos digitais para o ambiente educacional de forma planejada e
criteriosa, surgirão boas estratégias de ensino-aprendizagem e desenvolvimento de diversas
habilidades e competências num contexto disciplinar e transdisciplinar."(MENEZES, 2003, 
p. 2). 


Trabalhando o tema, educação e conhecimento no trânsito em um jogo, temos uma forma de melhorar  
o trânsito das cidades de maneira direta e descontraída, sem métodos de ensino massantes e
que muitas vezes acabam tirando o interesse dos alunos em aprender.
Visando esses e outros benefícios possíveis, decidimos criar o jogo Pacific Transit.\\

% ----------------------------------------------------------
% Seção de explicações
% ----------------------------------------------------------
\section{Um jogo educativo sobre trânsito}

Com o jogo Pacific Transit temos como principal objetivo melhorar a conduta do motorista 
em trânsito e ensinar um pouco sobre a legislação brasileira de tráfego, assim reduzindo a 
violência, negligência  e maus hábitos no trânsito. 

Dados do Ministério da Justiça divulgados
em 2011 mostram aumento de 32,4\% nas
mortes de jovens em decorrência de acidentes
no trânsito entre 1998 e 2008. No total, são
registradas por volta de 40 mil mortes por ano , e o Brasil está 
entre os líderes de mortes em acidentes de trânsito.

Entre as principais causas estão negligência (desatenção ou falta de cuidado ao realizar um ato), 
imprudência (má fé: velocidade excessiva, dirigir sob efeito de álcool, falar ao celular, desrespeitar sinalização, etc.) e por fim a 
imperícia (falta de técnica ou de conhecimento para realizar uma ação de forma segura e adequada).

\section{Metodologia}

O jogo educativo Pacific Transit tem como sua base a linguagem C pura, no padrão standard 99, 
aliada à biblioteca para jogos Allegro 5.
Essa biblioteca tem como objetivo a independência de plataforma de operação, ou seja, 
atuar em qualquer plataforma sem que precise ser alterado algum código. Assim sua portabilidade se torna alta,
o que é muito relevante principalmente quando trata-se de jogos.

Iniciamos o projeto com uma pesquisa sobre alguns dos principais problemas do trânsito brasileiro.
Com esses problemas listados, o próximo passo foi pensar em algumas ideias para o desenvolvimento do 
jogo que pudessem ajudar a resolver ou minimizar os reais problemas do trânsito.
Depois de definir a mecânica do jogo, partimos para o seu  desenvolvimento, em código, junto a uma pesquisa sobre a biblioteca 
Allegro 5 e design estrutural de um jogo (figura 3),para conseguirmos iniciar
a contrução do nosso jogo.

Como ferramentas de desenvolvimento do projeto, foram utilizados uma máquina virtual Ubuntu 12.04, o 
editor de texto Sublime text 2, compilador Gcc e o sistema de versionamento Git , aliado ao Github.
Trabalhando com o sistema de versionamento o grupo pode organizar melhor seus códigos,
procurar erros e sincronizar seus repositórios locais com maior facilidade,
agilizando o desenvolvimento do projeto.

\section{Resultados}

O jogo funciona da seguinte forma: o carro se move para frente automaticamente, 
e o jogador tem apenas o controle de movimentar o carro entre as pistas desviando ou acertando os 
objetos propositalmente.

Entre os possíveis objetos na pista, podemos ter placas de trânsito verdadeiras, placas de trânsito falsas e 
ações proibidas ao motorista, como beber e dirigir ao telefone, representadas respectivamente por um objeto 
ilustrativo de cerveja e celular.

Tendo ciência dos objetos apresentados o jogador tem, como meta, diferenciar as placas verdadeiras das
falsas e pegá - las para adquirir pontos e, ao mesmo tempo, desviar dos objetos que representam bebida 
ou algum tipo de ligação telefônica no trânsito.
Qualquer erro cometido pelo jogador é penalizado com perda de pontos ou até mesmo perder o jogo, pois,
fazendo uma analogia com o trânsito real, qualquer erro pode ser muito perigoso


\section{Considerações Finais}

De um modo geral, podemos concluir, que o poder de jogos interativos aliados a algum 
tipo de aprendizado é muito interessante, ainda mais quando podemos conciliar seu tema com algo 
benéfico a sociedade.

Ao final do desenvolvimento do projeto, observamos que, as limitações técnicas impostas pela
proposta (linguagem C std99 e biblioteca Allegro5) conseguiram ser contornadas com ideias simples tais como 
como estilo de modo de jogo implantado, porém sem perder espaço para plataformas mais poderosas
uma vez que o objetivo do jogo é conseguir atingir seu propósito de caráter educativo.

Com um jogo simples e interativo como o Pacific Transit, conseguimos prender a atenção das pessoas de maneira involuntária,
fazendo com que elas cumpram as tarefas do jogo com entusiasmo e, sendo um jogo educativo, o aprendizado consegue ser  
passado de maneira simples e eficiente, contemplando um dos principais objetivos do jogo Pacific Transit.\\
\\
% ---
% Finaliza a parte no bookmark do PDF, para que se inicie o bookmark na raiz
% ---
\bookmarksetup{startatroot}% 
% ---


% ----------------------------------------------------------
% Referências bibliográficas

% ----------------------------------------------------------

\bibliographystyle{plain}
\begin{thebibliography}{1}

\bibitem{Flusser:Suk:93}
MENEZES, C. S. (Org.).
\newblock Informática Educativa II - Linguagens para
Representação do Conhecimento.
\newblock {\em Vitória: UFES, 2003 Fascículo usado em
cursos de graduação do NEAD/CREAD/UFES , p. 1-10.}

\bibitem{Meijster:Wilkinson:PAMI}
C.~ANDRÉ , B.DAVID , S.CARLA.
\newblock Uma Análise Comparativa entre Jogos Educativos
Visando a Criação de um Jogo para Educação Ambiental.
\newblock {\em Departamento de Ciências Exatas, Centro de Ciências Aplicadas e Educação
Universidade Federal da Paraíba (UFPB) , p.1-4.}


\bibitem{Hu:62}
GAME BLENDER. Guide To the GameLoop. Game Blender. 2013.
\newblock Disponível em $<$http://wiki.gameblender.org/index.php?title=Guide\textunderscore To \textunderscore the \textunderscore GameLoop$>$
\newblock {\em Acesso em 18 de novembro de 2013}

\bibitem{Hu:62}
MONTEIRO,A. Brasil está entre os líderes de mortes em acidentes de trânsito. Folha. 2012.
\newblock Disponível em $<$http://www1.folha.uol.com.br/cotidiano/1085715-brasil-esta-entre-os-lideres-de-mortes-em-acidentes-de-transito.shtml $>$
\newblock {\em Acesso em 18 de novembro de 2013}

\bibitem{Hu:62}
G1 GLOBO.Brasil tem 43 mil vítimas de acidentes de trânsito por ano. G1 Globo. 2012. 
\newblock Disponível em $<$http://g1.globo.com/bom-dia-brasil/noticia/2012/09/numero-de-vitimas-de-acidentes-com-motos-aumenta-14-em-cinco-anos.html$>$
\newblock {\em Acesso em 18 de novembro de 2013}

\end{thebibliography}


\end{document}
